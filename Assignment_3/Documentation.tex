%\documentclass{article}
\documentclass{scrartcl}
\usepackage[utf8]{inputenc}
\usepackage[bottom]{footmisc}
\usepackage{amsmath,amsthm,amssymb}
\usepackage{algorithmic}
\usepackage{algorithm}
\usepackage{xcolor}
\usepackage[figuresleft]{rotating}
\usepackage{listings}
\usepackage{enumitem}
\usepackage{graphicx}
\usepackage[colorlinks = true,
            linkcolor = black,
            urlcolor  = blue,
            citecolor = black,
            anchorcolor = blue]{hyperref}
\setlength\parindent{0pt}

\def\imp{\rightarrow}

\newenvironment{level}%
{\addtolength{\itemindent}{2em}}%
{\addtolength{\itemindent}{-2em}}


\newtheorem{theorem}{Theorem}[section]
\newtheorem{corollary}{Corollary}[theorem]
\newtheorem{lemma}[theorem]{Lemma}



\makeatletter
\newcommand{\inlineitem}[1][]{%
\ifnum\enit@type=\tw@
    {\descriptionlabel{#1}}
  \hspace{\labelsep}%
\else
  \ifnum\enit@type=\z@
       \refstepcounter{\@listctr}\fi
    \quad\@itemlabel\hspace{\labelsep}%
\fi}
\makeatother

\definecolor{mGreen}{rgb}{0,0.6,0}
\definecolor{mGray}{rgb}{0.5,0.5,0.5}
\definecolor{backgroundColour}{rgb}{0.95,0.95,0.92}

\lstset{
		escapeinside={(*@}{@*)},
    backgroundcolor=\color{backgroundColour},   
    commentstyle=\color{mGreen},
    keywordstyle=\color{blue},
    numberstyle=\tiny\color{mGray},
    stringstyle=\color{gray},
    basicstyle=\footnotesize,
    breakatwhitespace=false,         
    breaklines=true,                 
    captionpos=b,                    
    keepspaces=true,                 
    numbers=left,                    
    numbersep=5pt,                  
    showspaces=false,                
    showstringspaces=false,
    showtabs=false,                  
    tabsize=2,
    language=C
}


\title{POP SS19 Homework 3 Documentation}
\subtitle{Robert Ernstbrunner, 01403753}

\begin{document}
\maketitle

\section{Introduction}
Assignment 3 builds on the previous Assignment, i.e., we had to extend our previous project in order to perform transformations on our \textit{AST}. First we had to find points in the \textit{AST} where two of the following three transformations are possible
\begin{itemize}
\item Loop Interchange
\item Loop Normalization
\item Induction variable substitution, i.e., 
\begin{equation*}
	v=v+c \iff v=\%vinit+c*I
\end{equation*} 
\end{itemize}


I picked \textit{LOOP INTERCHANGE} and \textit{LOOP NORMALIZATION} which I will explain further. I will also address strategies for the transformation process below.

\section{Loop Interchange}
Here, loop interchange is performed on two perfectly nested loops where the inner loop and the outer loop are simply switched. Loop interchange was a fairly easy transformation. A graphical representation can be seen in Figure~\ref{fig:LI}. My \textit{FOR} statements are structured in the following way. the right subtree of the \textit{FOR} node is a representation of the loop body. The left subtree has two special nodes. The \textit{INIT / COND} node contains the initialization and termination condition of the \textit{FOR} loop. This node is also the left subtree of the \textit{STEPSIZE} node which in its right subtree contains the stepsize of the \textit{FOR} loop. The root in Figure~\ref{fig:LI} is a \textit{FOR} node. In this case we only have to check if the right subtree is a \textit{FOR} node as well and then switch the left pointers (i.e., their initializations, conditions and stepsizes) for both \textit{FOR} nodes.

\begin{figure}
\centering
\includegraphics[width=\textwidth]{LI_none}
\vspace{.5cm}\\
\includegraphics[width=\textwidth]{LI}
\caption{Loop interchange of two perfectly nested loops. The top tree is the initial tree, the bottom tree is the result after the transformation.}
\label{fig:LI}
\end{figure}

\section{Loop Normalization}
Loop normalization normalizes the loop by transforming both the initial start variable and stepsize to $1$. This necessitates transformations of the loop variable inside the loop body as well as a back transformation after the loop.\\
This task was generally more tricky than loop interchange. Since loop normalization necessitates that the loop variable is transformed back after the loop, I had to insert an additional node that points to the \textit{FOR} statement on the left and to the 'back-transformation' statement of the loop variable to the right. This can be observed in the example in Figure~\ref{fig:LN}.\\
Notice that the normalization transformation makes loop interchange of two perfectly nested loops impossible since both \textit{FOR} nodes will no longer be directly connected after the transformation. Thus, if desired, loop interchange must be done before loop normalization. The root in Figure~\ref{fig:LN} is labeled as \textit{STMTLIST} (i.e., statement list). The left subtree \textbf{contains internal declarations} only and the right subtree points to the \textit{FOR} loop that is normalized. After the transformation the loop initialization variable is set to $1$, the upper bound is adapted and works correctly iff $(u - l + s) / s$ (where $u$ is the upper bound, $l$ is the start value and $s$ is the stepsize) is integral and the stepsize is $1$ as well. The loop variable in the right subtree was supplemented as well.

\section{The transformation process}
The first thing that the parser does is to add the include statement for the \textit{stdio.h} header. This little measure gets rid of the \texttt{printf} compiler warning for the transformed file. Then the lexer ECHOS everything until a "$\{$" bracket has been found. This must be the first bracket of the main function or else the program will not work. Since my \textit{AST} is capable of handling declarations I let it handle the internal declarations as well. After the \textit{AST} transformation I call a print function that handles round and square brackets and declarations, skips "STMT" nodes and treats "FOR" statements the right way.
After everything is printed out, the Makefile makes sure that the output is redirected in the output file.
\begin{figure}
\centering
\includegraphics[width=\textwidth]{LN_none}
\vspace{.2cm}\\
\includegraphics[width=\textwidth]{LN}
\caption{Loop normalization of a for loop with declarations. The top tree is the initial tree, the bottom tree is the result after the transformation.}
\label{fig:LN}
\end{figure}

\section{How to run}
This is pretty straight forward. I've included a \textit{test\_function.c} file that is slightly adapted to the example given on the Assignment. However, there is one thing I'd like to point out again. I shifted the file generation task from the \textit{parser}/\textit{intermediate.c} to the \textit{Makefile} by printing and channeling the results to the file \textit{$<$filename$>$\_out.c}. \\
In Table~\ref{tab:commands} are the ordered (most basic) commands to perform the transformations and execution.

\begin{figure}
\centering
\begin{tabular}{|c|l|p{7cm}|}
\hline
 & \textbf{command} & \textbf{description} \\
\hline
\hline
1. & \texttt{make} & build the parser \\
\hline
2. & \texttt{make FILE=$<$filename$>$ transform} & parse and transform the file \textit{$<$filename$>$.c} and store the result in \textit{$<$filename$>$\_out.c}\\
\hline
3. & \texttt{make FILE=$<$filename$>$ run} & compile and run \textit{$<$filename$>$.c} and \textit{$<$filename$>$\_out.c} for result comparison.\\
\hline
 & \texttt{make png} \textit{(optional)} & generates the dot- and .png file for the abstract syntax tree of the most recently transformed input file. \\
\hline
\end{tabular}
\captionof{table}{Type \texttt{make help} for a complete list of commands.}
\label{tab:commands}
\end{figure}

\section{Code for the ASTs from Figures~\ref{fig:LI} and \ref{fig:LN}}
Figures~\ref{code:LI} and \ref{code:LN} show the input and the generated code.\\
The code transformation produces some additional semicolons. This is a thing that could be optimized, but unfortunately this becomes surprisingly messy fast.

\begin{figure}
\centering
\textbf{INPUT}:
\begin{lstlisting}
int main()
{
	for(i=0; i<10; i++)
		for(j=1; j<=20; j=j+2)
			B[i][j]=i*j;
}
\end{lstlisting}
\textbf{OUTPUT}:
\begin{lstlisting}
/* performed loop_interchange. */
#include <stdio.h>

int main()
{
for(j=1; j<=20; j=j+2)
{
for(i=0; i<10; i++)
{
B[i][j]=i*j;
}
;
}
;
}
\end{lstlisting}
\caption{The example code and transformed code after loop interchange.}
\label{code:LI}
\end{figure}

\begin{figure}
\centering
\textbf{INPUT}:
\begin{lstlisting}
int main()
{
	int i,k;
	for (i=10; i<=20; i=i+2)
		C[i]=i * k;
}
\end{lstlisting}
\textbf{OUTPUT}:
\begin{lstlisting}
/* performed loop_normalization. */
#include <stdio.h>

int main()
{
int i,k;
for(i=1; i<=6; i=i+1)
{
C[(10+(i-1)*2)]=(10+(i-1)*2)*k;
}
;
i=10+(i-1)*2;
;
}
\end{lstlisting}
\caption{The example code and transformed code after loop interchange.}
\label{code:LN}
\end{figure}

\section{AST and code for a full transformation on test\_function.c}
Figures~\ref{fig:FULL}, \ref{code:FULL_none} and \ref{code:FULL} depict the initial, more complex input that is transformed first by loop interchange and then by loop normalization.

\begin{sidewaysfigure}
\centering
\includegraphics[width=\textwidth]{FULL_none}
\vspace{.2cm}\\
\includegraphics[width=\textwidth]{FULL}
\caption{Loop interchange with consecutive loop normalization of a more complex example. The top tree is the initial tree, the bottom tree is the result after the transformations.}
\label{fig:FULL}
\end{sidewaysfigure}


\begin{figure}
\centering
\textbf{INPUT}:
\begin{lstlisting}
/* declarations outside */
int A[100];
double C[100];
int B[100][100];
int main()
{

	/* declarations inside */
	int i,j;
	int v;
	double k;

	k = 3.1415;
	/* now only executable statements ... */

	/* A. loop normalization */
	for (i=10; i<=20; i=i+2)
		C[i]=i * k;
	/* output result */
	printf("i, C[20]: %d, %.2f \n", i, C[20]);
	for (i=0; i<=20; i++) printf("%.2f ", C[i]);
	printf("\n");

	/* C. loop interchange */
	for (i=0; i<10; i++) 
		for (j=1; j<=	20; j=j+2)
			B[i][j]=i*j;

	printf("i,j, B[i][j]: %d, %d, %d \n", i, j, B[i][j]);
	for (i=5; i<8; i++) {
		printf("i=%d : ", i); /* prevents loop interchange */
		for (j=1; j<=20; j++)
			printf("%d ", B[i][j]);
		
		printf("\n");
	}

}
\end{lstlisting}
\caption{The test\_function.c code before the transformations.}
\label{code:FULL_none}
\end{figure}

\begin{figure}
\centering
\textbf{OUTPUT}:
\begin{lstlisting}
/* performed loop_interchange. */
/* performed loop_normalization. */
#include <stdio.h>


int A[100];
double C[100];
int B[100][100];
int main()
{
int i,j;
 int v;
;
 double k;
;
k=3.141500;
for(i=1; i<=6; i=i+1)
{
C[(10+(i-1)*2)]=(10+(i-1)*2)*k;
}
;
i=10+(i-1)*2;
;
printf("i, C[20]: %d, %.2f \n",i,C[20]);
for(i=1; i<=21; i++)
{
printf("%.2f ",C[(0+(i-1)*1)]);
}
;
i=0+(i-1)*1;
;
printf("\n");
for(j=1; j<=10; j=j+1)
{
for(i=1; i<11; i++)
{
B[(0+(i-1)*1)][(1+(j-1)*2)]=(0+(i-1)*1)*(1+(j-1)*2);
}
									(*@ \vdots @*)                                 (*@ \vdots @*)
}
;
i=5+(i-1)*1;
;
}
\end{lstlisting}
\caption{Snippets of the test\_function.c code after the transformations.}
\label{code:FULL}
\end{figure}

\bibliographystyle{plain}
\bibliography{Documentation}
\end{document}
